\section{Related work}

\subsection{Reducing Capacitance Requirement}
The need for reducing the energy consumption needed for power-loss protection
arises in different contexts. A few studies reduce the total energy consumption
by speeding up the back-up process at a power failure using the fast media. Guo
et al. reduce the capacitance requirement by writing back the volatile buffer
data into PRAM (Phase Change Random Access Memory), which is faster and uses
lower power than NAND flash~\cite{GuoYZC13date}. They argue that this reduction
enables to replace the supercapacitors that are suffering from serious aging
problems with the regular capacitors, which have more reliable
characteristics~\cite{huang2011life}. This consequently enhances the robustness
of storage device.  As a similar approach, Smartbackup~\cite{HuangWQLS15hpcc}
proposes dynamic NAND channel allocation and SLC (single-level cell) mode
programs to make the dump process shorter at sudden power-off. It makes full
use of available SSD channels and dynamically adjusts these channels based on
the available power of the capacitor to exploit the nature of high parallelism
on NAND flash arrays.  In addition, as the SLC mode program shows significantly
shorter time than subsequent MLC, TLC, or QLC mode, it programs the target page
to dump in SLC mode to achieve shorter time required for dumping process.
%However, because of inherent trade-off imposed by SLC mode program, the size of
%effective storage space decreases and this causes additional overhead for space
%management.

\begin{table}[tb]
    \centering
    \fontsize{11}{11}
    \small
	% Model Manufacturer Category PLP-Support Capacitor 
    %\begin{tabular}{p{2.4cm}|p{1.3cm}|p{1.2cm}|l}
    %\begin{tabular}{p{2.4cm}|p{2.4cm}|p{2.4cm}|p{2.4cm}|p{2.4cm}}
    \begin{tabular}{|p{5cm}|l|}
        % \hline
        % \multirow{4}{*}{{\rotatebox{90}{\parbox{1.2cm}{\centering \footnotesize{Concurrency}}}}} 
		\hline
		\bf{Configuration} & \bf{Size} \\ \hline \hline
        Channel & 8x \\ \hline
        Way & 4x \\ \hline
        Die & 4x \\ \hline
        Plane & 4x \\ \hline
        Page Size & 4KB \\ \hline
        SSD Capacity & 512GB \\ \hline
    \end{tabular}
    \caption{\textbf{SSD Configuration.}}
    \label{tab:ssd_config}
    % \vspace{-10pt}
\end{table}

\begin{table}[tb]
    \centering
    \fontsize{11}{11}
    \small
	% Model Manufacturer Category PLP-Support Capacitor 
    %\begin{tabular}{p{2.4cm}|p{1.3cm}|p{1.2cm}|l}
    %\begin{tabular}{p{2.4cm}|p{2.4cm}|p{2.4cm}|p{2.4cm}|p{2.4cm}}
    \begin{tabular}{|p{5cm}|l|}
        % \hline
        % \multirow{4}{*}{{\rotatebox{90}{\parbox{1.2cm}{\centering \footnotesize{Concurrency}}}}} 
		\hline
        \bf{Data Type} &  \bf{Size} \\ \hline \hline
        % \footnotesize{\bf{Ratio}} \\ \hline \hline
	    {User Data Buffer} & {4MB} \\ \hline
		{Mapping Table} & {512MB} \\ \hline
		{Mapping Table Directory} & {512KB} \\ \hline 
% 		\footnotesize{Mapping Table Directory} & \footnotesize{128KB} & \footnotesize{0.02}\% \\ \hline
		{Metadata for Allocation} & {1MB} \\ \hline 
		{Metadata for GC(Garbage Collection)} & {9MB}  \\ \hline 
		{Total Buffer Memory} & {526.5MB} \\ \hline
% 		\footnotesize{SSD Capacity} & \footnotesize{512GB} & -  \\ \hline
    \end{tabular}
    \caption{\textbf{Components of the SSD-internal buffer.}}
    \label{tab:ssd_buff_comp}
    \vspace{-10pt}
\end{table}

Another approach to reducing the capacitor size is protecting a part of the 
volatile buffer. DRWB (Dual-Region Write Buffer) divides the internal-SSD
buffer into small protected region (backed by a capacitor) and large
unprotected region and when the data on unprotected region is updated, the 
delta for the page is logged in the protected region~\cite{KimK15sac}.  With
this differential logging, DRWB logically realizes the non-volatile buffer
using a small size of capacitor. However, the proposed technique only regards
the user data, having no consideration on the metadata such as mapping table, 
despite that it actually accounts for most of the internal buffer of SSDs.  Furthermore,
commercial SSDs typically do not cache read data in the buffer because the host
memory can serve as a cache memory of the storage device. For these reasons,
the effectiveness of DRWB may be limited in practical environment. 

In line with this, Kang et al. present an SSD prototype with durable cache,
called DuraSSD, to enhance the write performance in database and NoSQL systems.
They observe that frequent cache flushing of SSD which is requested to
guarantee the atomicity and durability of transactions, makes a long write
latency and imposes a serious performance degradation~\cite{KangLMKO14sigmod}. To
resolve the problem, they maintain the internal-SSD cache
durable by using the partially protected DRAM with a small size of tantalum
capacitors. DuraSSD maintains a group of user data pages and a page mapping
table in DRAM cache; on the power-failure, it flushes all of the user data and
dirty mapping entries into the dump area in flash memory, because flushing
the entire mapping table requires excessive time and energy.  On recovery,
Dura-SSD re-writes the user data in dump area to their permanent location in
NAND flash with mapping table updates, and it merges dirty mapping entries with
its permanent copy.

Spartan-SSD shares similarities with DuraSSD in that they both selectively
protect the user data, but as shown in the performance evaluation, it has
notable differences. When the working-set size goes beyond the protected buffer
size by capacitors, DuraSSD incurs a serious performance decrease, while
Spartan-SSD invariantly provides excellent performance across the various
workloads, even without any additional capacitors.


\subsection{Maintaining Capacitor Aging}
Another group of studies attack the capacitor aging problem.  Alcicek et al.
demonstrate the ultracapacitor aging according to the temperature through
experimental measurements~\cite{alcicek2007experimental} and Hannonen
et al. present a method to detect the capacitor degradation using the
variations of the output voltage at the dc-dc converter~\cite{TIA2016}. Gao et
al. detect the current available capacitence and bound the number of dirty
pages under the limit so as to prevent a data loss against the capacitor
aging~\cite{GaoSDLXS18glvlsi, GaoSLLXYZ19tcad}. They run a periodic background
write-back process to detect the dirty page budget dynamically, and activate
the write-back process when the number of dirty pages approaches the budget
closely. Spartan-SSD assumes an overly-charged capacitor enough to provide a
sufficient capacitence during the SSD lifetime, which is like a commercial SSD.


\subsection{Write Buffer in Scalable Storage}
%% 은: 고용량 SSD 에서 제한된 write buffer 를 효율적으로 활용하고자 하는 연구. 
%% RFLUSH 는 고용량 SSD 에서 write buffer 가 커질 거라는 거. 
Some studies explore ways of using the internal write buffer efficiently in
scalable SSDs. Chen et al. project that even the high capacity of SSDs will use
the small size of write buffer because the capacitor that protects the buffer
does not scale well due to the cost, size, and reliability
constraints~\cite{ChenLZ19tc}. Nevertheless, they observe that the small sized
write buffer can be effective for reducing write traffic in particular
applications that perform journaling heavily. Motivated by this observation,
they present the application-SSD co-design to reduce the data writes buffered
for heavy logging/journaling applications. They propose to protect write-hot
log/journal data with capacitors while the log/journal data being durable.  In
addition, they propose NVMe interface extension for host to notify SSDs the
ranges of write-hot LBAs for more efficient protection by capacitors with
reduced complexity of hot/cold separation.  It reduced substantial amount of
flash memory write traffic with few megabytes of capacitor-powered write
buffer, but it is specific to heavy log/journal applications and requires
change of application code to benefit from its scheme.

