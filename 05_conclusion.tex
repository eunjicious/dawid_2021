\section{Conclusion}
This paper presented a novel SSD design called \ours{}.
\ours{} protects a fraction of the storage-internal buffer to overcome capacitance constraints
in high-capacity SSDs. \ours{} maintains performance by reducing a dirty memory footprint 
of in-DRAM data through the cost-effective re-ordering, which underlies the increasing queue depth of the storage interfaces. 
%We implemented a \ours{}-SSD prototype in FEMU, an open-source SSD
%development framework. 
Performance evaluation with various workloads shows that \ours{} offers 
up to 42.7\% higher IOPS and 49\% less write traffic when the capacitance is highly limited.
% delivers only XX\%-XX\% performance slowdown when capacitance is reduced to 1\%, while conventional SSD using FIFO decreases performance by XX\%.

\iffalse
In this paper, we raised an issue about capacitance constraints in scalable
SSDs and presented a novel SSD design called \ours{} to overcome the
limitation.  \ours{}-SSD protects a part of the buffer, but reduces the dirty
memory footprint by exploiting the increasing queue depth of the storage
interfaces. We implemented a \ours{}-SSD prototype in FEMU, an open-source SSD
development framework. Performance evaluation using the prototype shows that \ours{}-SSD 
delivers only XX\%-XX\% performance slowdown when capacitance is reduced to 1\%, 
while conventional SSD using FIFO decreases performance by XX\%.
\fi

\iffalse
Hexa's design, implementation, and evaluation to
reduce performance overhead due to metadata flush in a situation where PLP is
partially supported on enterprise-class SSDs.  Our desing operates using only a
small amount of Capacitor's capacity compared to the previous one. It also
minimizes the impact of metadata by buffering requests using the scalability of
the write buffer.  Our evaluation results show that Hexa improves performance
in situations where requests are randomly generated.  In addition, JESD and
real-benchmark results comfirm that there are advantages in a real-world
environment.
\fi
